%
%   NI SHI MU RA
%

\documentclass[10pt,b5paper,papersize,dvipdfmx]{jsbook}
% \documentclass[11pt,b5paper,papersize,dvipdfmx,draft]{jsbook}

\usepackage{vuccaken}
\usepackage{vuccaken2019}

\begin{document}

% - - - - - - - - - - - - - - - - - - - - - - - - - - -
\kaishititle{音とサインとそれからイヤホン♪}{物理科学科3回生}{西村宗悟}
% - - - - - - - - - - - - - - - - - - - - - - - - - - -

%
\section{はじめに}
とりあえず作成。熟成したらtemplateの方で上書きする。

\section{a}\section{a}\section{a}
a
\section{a}\section{a}\section{a}\section{a}\section{a}\section{a}\section{a}\section{a}\section{a}\section{a}\section{a}\section{a}\section{うんこ}
\section{a}
こんにちは。今年も会誌頑張って作ろうね。
今年からはGitHubで管理するよ!


% 参考文献
\sanko
\begin{enumerate}
  \item 青木直史 (2014), 『ゼロからはじめる音響工学』,講談社.
  \item 久保和宏ほか(2009), 『音響学ABC』, 技報堂出版.
\end{enumerate}


\end{document}
%
% ファイトだよ!
%