%
%     奥付
%

\documentclass[10pt,b5paper,papersize,dvipdfmx]{jsbook}

\usepackage{vuccaken}
\usepackage{vuccaken2019}

% --------------------------------------
\begin{document}

% \thispagestyle{empty}

\markboth{}{} % headerのchaper nameとsection nameを消す
\clearpage % 右開きじゃなくて良い

\noindent
{\large \bfseries 物理科学研究会の略歴}
\begin{table}[H]
  \begin{tabular}{ll}
    1949年 & 核物理研究会として発足 \\
    1973年 & 現存する最古の会誌 出版 \\
    2000年 & 物理科学研究会に改名   \\
    2016年 & 会誌「白夜 第一号」出版 \\
    2017年 & OB会の開催           \\
          & 会誌「白夜 第二号」出版 \\
    2018年 & 会誌「白夜 第三号」出版 \\
    2019年 & 会誌「白夜 第四号」出版 \\
          & OB会の開催(予定)
  \end{tabular}
\end{table}

\vfill

{\bfseries 令和元年度 物理科学研究会誌}\par
{\large \bfseries 白夜 第四号}\vspace{-1zw}\\
\hrulefill \par
2019年12月1日\hspace{2zw}初版発行\par
% 2018年11月30日 \quad 第2版発行

\vspace{1.5zw}

\begin{minipage}{0.15\hsize}
    % \vspace{16.5zw}
    \hspace{1zw}
\end{minipage}
\begin{minipage}{0.7\hsize}
  \begin{itemize}
    \item[{\bfseries 編集}:] 中山~敦貴
    \item[{\bfseries 表紙イラスト}:] 西村~宗悟
    % \item[{\bf 裏表紙イラスト}:] 西村~宗悟
    % \item[{\bf 著者}:] 立命館大学 物理科学研究会
    \item[{\bfseries 発行所}:] 立命館大学 物理科学研究会
    \item[ ] 〒525-8577 滋賀県 草津市 野路東 1-1-1
    \item[ ] 立命館大学 BKC アクトα サークルルーム6
    \item[ ] メール: \url{vuccaken@gmail.com}
    \item[ ] ホームページ: \url{vuccaken.github.io}
    \item[ ] Twitter: \url{@vuccaken}
  \end{itemize}
\end{minipage}

\par \vspace{0.5zw}
\hrulefill \par
% Printed in Japan

\end{document}

%
% お疲れさまです
%