%
%   TEMPLATE
%   コピペ用
%

\documentclass[10pt,b5paper,papersize,dvipdfmx]{jsbook}

\usepackage{vuccaken}
\usepackage{vuccaken2019}

% スタイルファイルの読み込みや自作マクロは、
% 最終的には vuccaken2019.sty の中に書いてください。
% とりあえずはここに書いてもらって構いません。


\begin{document} % - - - - - - - 以下本文 - - - - - - - - - -

\mokuji{2} % 目次出力

% - - - - - - - - - - - - - - - - - - - - - - - - %
\kaishititle%
  {宇宙の始まりはどこ?~ホーキング博士の活躍と理論~}% title
  {物理科学科1回生}% 所属
  {ふじき}% name
% - - - - - - - - - - - - - - - - - - - - - - - - %

%
\section*{はじめに}
今年度から物理科学研究会に入部した物理科1回生の藤木です。\par
今回の会誌では自分は宇宙についてホーキング博士の考えを紹介しながら書いていこうと思っています。\par
今回が初めての会誌であるのに加えて、ほとんどが文章でかなり読みずらいかもしれませんが、興味のあるセクションだけでも目を通してもらえたら幸いです。\par




\newpage
%
\section{宇宙に関する疑問}
皆様は宇宙に対する疑問といわれるとどのようなものを思い浮かべるでしょうか。\par
たとえば、「宇宙に果てはありますか?」「宇宙に始まりはあったのですか?」「宇宙人はいますか?」などといったものが思い浮かぶかなとおもいます。\par
このようなぱっと思いつく一見単純で素朴な疑問も、今まさに宇宙論研究者が現在進行形で取り組んでいるテーマに直結しているものもあります。\par
ちなみに、その疑問に答えるとするなら、(それが完全に正しいと言い切ることはできませんが)\par
「宇宙に果てはなく、体積はほぼ無限といえるほど大きいです。」「うちゅうは今から約138億年前に誕生したと考えられており、その意味では、この宇宙は、無限に広がった空間と、有限な過去と無限に続く未来とからなりたっています。」「直接検証できる可能性が低いとはいえ、この宇宙のどこかには地球と同様、生命、さらには知的生命体が存在しているはずです。」と言えるでしょう。\par
こんかいは、先に挙げた疑問にもあった「宇宙に始まりはあったのですか?」というものについて書いていきます。\par



\newpage


\section{ ホーキング博士について}

\subsection{ホーキング博士}
ホーキング博士は記憶に新しい2018年3月14日に享年76歳でこの世を去ったイギリスの理論物理学者で、
本名はスティーブン・ウィリアム・ホーキングといい、彼は生前その業績とALS(筋萎縮性側索硬化症)
を患って車椅子に乗っていたことから「車椅子のニュートン」と呼ばれていました。\par
ホーキング博士の業績については後のセクションでふれるのでこのセクションではホーキング博士の人となりについて書いていこうと思います。\par

\subsection{ホーキングの生い立ち}
彼は1942年1月8日にイギリスのオックスフォードに生まれました。当時、イギリスは第二次世界大戦のまっただなかで、
彼の両親はロンドン北部から疎開してオックスフォードに暮らしていたそうです。
彼が8歳の時、一家はロンドンの北にあるセント・アルバンズへ引っ越し、ホーキングはそこにある学校に通い始めました。\par
その後、父親と同じカレッジであるオックスフォード大学のユニバーシティー・カレッジに進学し、物理学を専攻したホーキングは1962年に
「自然科学」で学士号を取得します。\par
オックスフォード大学を卒業した彼はケンブリッジ大学の博士課程に進み、1965年、宇宙論の研究で博士号をもらいました。そして、その後もケンブリッジ大学に
研究フェローとして残り、アインシュタインの重力理論と量子論の研究を続けました。\par
1974年には史上最年少で王立協会のフェローに選出され、1977年にはケンブリッジ大学の重力理論講座の教授に就任されます。さらに彼は1979年には
37歳でかつてニュートンが在籍したというルーカス職数学教授に就任します。\par
しかし、このような華々しい経歴の裏でホーキングは21歳で筋萎縮性側索硬化症(ALS: Amyotrophic Lateral Sclerosis)の診断を受けており、それ以来“
不治の病”と戦い続けることになってしまいました。\par
この筋萎縮性側索硬化症は治療のための有効な治療法は現在も確立されていない難病で、病気が進むに従って、手足をはじめ、体が自由に動かせなくなり、やがて話すこと、食べることはおろか、呼吸することも難しくなっていきます。\par
そんな難病を患って余命いくばくもないといわれてもなおホーキングは絶望するのではなく前へ進み続けました。
\newpage

\subsection{病を得てすべてがボーナスになる}
ALSという難病だと診断されても絶望せず、前に進み続ける精神を与えたのは、病院で目撃した白血病の少年だったといいます。\par
自伝でも多くを語ってはいませんが、余命いくばくもないと知らされた衝撃のなか、ホーキングは彼の死を目撃しました。
自分は最悪の運命だと思っていたが、そうではなかった。自分よりもっと不幸な人がいる。
そんなふうに思い直して、ホーキングは自分を憐れみたくなりそうになったときは、いつもその少年のことを思い出して、
自分に鞭を入れていたのでした。\par
「死刑宣告を受けたが、まだ執行猶予期間中。その間に、少しでも善いことをしたい」\,
というのがホーキングの願いだったようです。\par
ホーキングにとって、ALSを宣告されて以来、すべての時間は「ボーナス」でした。いつ死んでもおかしくないという身であれば、
人生に期待は禁物でしょう。そんな徹底した悲観論から生まれる今日という日へのボーナス感。それが50年続いたからこそ、ホーキングは思考において
誰よりもエネルギッシュに、遠くへ行くことができたのかもしれません。\par
もし私たちがホーキングと同じようにすべてをラッキーに感じて生きることができたら。
ホーキングの業績は、人間のポテンシャルを独特のかたちでみせてくれているものだ、とも言えるのではないでしょうか。\par
彼は生前、2004年12月12日ニューヨークタイムズにこのような言葉を残しています。\par
\begin{quote}
「My expectations were reduced to zero when I was 21. Everything since then has been a bonus.」\par
(私の期待は21歳の時にゼロまで落とされた。それ以来、すべてが私にとってボーナスなんだ。)\par
\end{quote}

\subsection{宇宙に「夢」を語らない実証論者}
宇宙論者というと、頭のいいロマンチストというイメージもあるかもしれませんが、ホーキングは違います。彼は、徹底的な実証論者であり、無神論者でもありました。\par
この実証論者とは、現象が起こった時に、何がどのように働いてその現象を起こしたのかについて考えずにその現象にのみ焦点を当て
「何が起きたのか」という結果にだけ関心を寄せる論者です。宇宙論者はざっくり言うとこの実証論者と実在論者に分かれていて、
後者の実在論者は反対に「現象の裏には何が存在するのか」に関心を寄せます。\par
\newpage

ホーキング博士はバリバリの実証論者で研究手法のスタンスとしてはかなりビジネスライクで、
結果だけを重要視して「モノで確かめたい」という衝動を抱くことなく進んでいきます。
考えても仕方ないものは考えない。現実的にいきましょうよ、と。理論的に整合しているなら、それで納得なのです。\par
また、彼は無神論者で後に紹介する無境界仮説において半ば神の存在を否定したことによって1981年に開かれたヴァチカン会議で当時の法王に「あんまり行き過ぎないように」とクギを刺されたといいます。(直接的にいわれたわけではありませんが。)\par
そんな無神論者であったホーキング博士にとっては「あの世」は存在しないわけで、彼は自分の存在をコンピューターに例えていて、死ぬ=脳が活動お停止することであり、つまりそれは壊れたコンピューターのようなものだ、と述べています。コンピューターは壊れたらそれっきりです。
自分も死んだら同じ、というのは、非常に潔い人生観と言えるでしょう。\par
人間誰しも死は不安なものですから、死んだあとも自分の居場所はあってほしい、天国があってほしい、というふうに考えたくなります。
自分が死んで、肉体が朽ち果てて、脳が機能を停止した後も、もしかしたら心とか魂とか呼ばれるものは残っていて、天国に行って安らぎを得られたら嬉しい、みたいに考える。いわば「死後の世界」を保険として引き合いに出しているのです。\par
しかし、ホーキングはそこを無神論者として、一切否定します。自分の死についてかなりドライに語っているのは、死そのものが、もうあまり怖くなくなっていたからではないかな、と想像します。ALSを患って以来、彼は他の誰よりも自らの死について深く頻繁に向き合ってきたはずです。
そのように自分の死や自分が死んだ後のことにさまざまな角度から考えを巡らせ尽くしていたからこそホーキング博士は動揺したり美化したりすることなく、死と向き合うことができていたのではないでしょうか。\par



\newpage

\section{ホーキング博士と宇宙}

\subsection{車椅子のニュートン}
先ほどのセクションでふれたように生前ホーキング博士は車椅子のニュートンと呼ばれていました。\par
そんな彼の科学者としての業績は、どのようなものだったのでしょうか?その答えは、まさにこの呼び名が如実に表しています。\par
つまり彼は、ニュートン級の物理学者でした。\par
ニュートンのように、物理学者たちの、そして人々の世界観を変えるほどの成果を挙げたのです。\par
では、アイザック・ニュートン(1642~1727)以前と以後では、物理学は何が違ったのでしょう。それは、ニュートン力学があったかなかったか、です。\par
ニュートンが登場するする以前にも力学に関する法則はあれこれ発見されていましたが、それらは個々の現象を説明する法則として、バラバラに存在しているにすぎませんでした。\par
例えば、ガリレオは真空状態であれば落下する物体の速度はどれも等しいという「落体の法則」を発見していましたし、ケプラーは、惑星の運動に関する「ケプラーの法則」を発見していました。ただし、それらは、特定の事象を説明するためだけに用いられる法則だったわけです。\par
ニュートンは、地上における落体の法則と宇宙における惑星の運動の法則が、実は同じ運動法則で説明できることを発見しました。それがニュートンの運動三法則と万有引力です。これにより、地上と宇宙は同じ法則で動いている、という認識が広まりました。\par
地球と宇宙は別ものだという人々の見方、考え方を完全に変えてしまったわけです。\par
それまで地上を支配する法則と天上界(宇宙はこう呼ばれていました)を支配する法則は違う、と人々は思っていましたが、実は同じなんだ、と。ガリレオとケプラーは同じことを言っているんだということが、ニュートン力学の確立によって示されたのです。\par
地上には地上の法則があり、天上界には天上界の法則がある。そういうかつて誰しも当たり前に感じていた世界観、宇宙観をがらりと変えてしまった。\par
「ニュートン級」の仕事といえば、このように人々のパラダイム(時代ごとの支配的な考え方・味方)を根っこから覆すような業績といえます。\par
そして、それほどまでにインパクトの大きな成果を挙げたのが、ホーキングである、というわけです。ホーキングもまた、宇宙論の分野で物理学者たちの物の見方を大きく変える業績を挙げました。\par

\newpage

ニュートンの次に現れた大物物理学者といえば、アルベルト・アインシュタイン(1879~1955)です。宇宙について考えるとき、通常、宇宙論学者たちは、アインシュタインの方程式を使ってあれこれ計算します。さまざまな条件を仮定し、アインシュタインの方程式で計算してみて予測を行うわけです。天体の動きや、星の一生、ブラックホールの実態、宇宙全体の様子や始まり、そして未来とか、そういう非常にスケールの大きなものを扱う宇宙論では、いわばそれが研究者たちの間では常識的なアプローチでした。\par
そんな宇宙論にホーキングは、なんと量子力学をもちこみました。\par
量子力学は、ミクロの世界における物質の振る舞いについて研究する分野ですから、広大な宇宙を扱うアインシュタイン方程式の世界とは、まるで正反対です。原子、素粒子といった非常にスケールの小さいものについて研究する方法を、非常に大きなものを研究する分野に持ち込んだ。本来なら水と油以上に性質の異なるふたつのツールを、ひとつの目的のために一緒にしてしまったわけです。\par
このような発想の根本が、地上界と天上界の法則を同一視したニュートンと同じなわけです。\par
宇宙はとても大きくて、今も膨張し続けている。ということは、時間を遡れば、宇宙はもともと小さかった。いえ、きわめて小さかった。その頃の宇宙の姿を計算するためには、アインシュタインの方程式と量子力学の方程式の両方が必要だとホーキングは考えました。そして、性質の異なるふたつの方法論を融合、統一したうえで、宇宙論を唱えました。\par
それによって、宇宙の始まりについて、いろいろなことがわかってきました。その業績を称えて、「車椅子のニュートン」と呼ばれているわけです。\par
このセクションでは、宇宙論と量子論を統一したことによってホーキングが見出した発見について紹介していきます。\par
 
\subsection{特異点定理}
ホーキング博士の発見について書いていく前に、「宇宙の始まり」に関する主張の変遷について、もう少し踏み込んで説明していきます。\par
アインシュタインは、等速直線運動だけを扱った特殊相対性理論において、時間は普遍的かつ絶対的な1つの速さで流れているのではなく、動いている状態によってその速さは異なることを示しました。さらに、重力の作用をも含めた一般相対性理論では、重力によって時間と空間(時空)は曲がる、ということを示しました。アインシュタインは、重力による時空の曲がり具合は、その物質の質量に比例することを方程式によって表したのです。\par
それはやがて観測によって確かめられることとなり、宇宙に存在する、天体をはじめとする物質により宇宙の時空(時間と空間のこと)は、その重力の影響を受けてあちこちで歪んでいる、という宇宙像が見えてきました。\par
ということは、宇宙を満たす物質による引力でやがて宇宙は収縮するはずですが、アインシュタインの目には、宇宙はまったく変化しない静的なものとして広がっていました。あるいは、宇宙とはそうあるべき、という彼の宇宙観がまずあった、と言ってもよいかもしれません。\par
アインシュタインは、「宇宙定数」と呼ばれる項を方程式に導入することで、収縮しようとする宇宙には、それに対抗して宇宙を押し広げようとする力も働いており、物質の間に働く万有引力と釣り合っている。だから宇宙は収縮も膨張もしないのだ、という宇宙モデルを提唱しました。\par
アインシュタインにとって、宇宙とは永遠不変であるべきだ、という美学があったのです。ホーキングはこの選択を理論物理学の歴史において、失策であると指摘しています。一方アインシュタインは、この判断は自身にとって最大の誤りであったかもしれないと振り返っています。\par
というのも、1920年代の段階で、宇宙は膨張していることが観測によって確認されたからです。\par
宇宙は膨張している。こうなると、宇宙を静的に保つために宇宙定数というものを考えることは、現実との整合性において具合がわるくなります。(ただし、ごく小さな値ながら、やっぱり宇宙定数はあるかもしれない、というのが最近の観測結果ですが、今は考えなくて大丈夫です。)\par
そして、宇宙が膨張しているということは、時間を巻き戻せば、うちゅうの星々はもっとぎゅっと密集していたに違いありません。さらに遡れば、その形はとても小さな1点にまで巻き戻せるのではないか。あるとき突然、現在のような宇宙が誕生したのではない限り、当然、そのように考えられます。\par
この宇宙をぎゅっと押し込めた1点を「特異点」といいます。ホーキングは、ロジャー・ペンネローズ(1931~)と共に、特異点が「ビックバン」と呼ばれる大爆発を起こして宇宙が始まったことを一般相対性理論を用いて証明することに成功しました。\par
これを「特異点定理」といい、それはホーキングにとっては、宇宙論の研究の原点にあたるものでした。\par
ホーキングは特異点定理によって、宇宙の最初は特異点であり、(原因は不明ですが)特異点のビックバンという大爆発から始まり、それが理由で宇宙が現在も膨張し続けていることを示しました。\par
ただし、定理には必ずその定理を成立させるための「前提条件」があります。ホーキングの特異点定理の場合、アインシュタインの一般相対性理論(重力理論)が正しく、さっきの話題に挙がったアインシュタインの導入した「宇宙定数」のような力(宇宙の収縮に反発する力)が宇宙に強く働いていないうえ、ぐるぐる回るような時間軸が存在しない必要があります。\par
しかし、この「宇宙定数」のような力が働くかどうかですが、近年の天文観測によれば、(確定したわけではありませんが)存在する可能性が出てきました。\par
こうなると、特異点定理は、不動の定理というわけにはいかなくなり、「重要な出来事」のひとつに過ぎないものとなってしまいました。\par
しかし、そんなことはホーキング自身も先刻承知で、特異点定理を発表した後、すぐにホーキングは、「アインシュタインの重力理論に量子的な効果を加味したらどうなるか」についての研究にとりかかったのでした。\par
その成果のひとつめがブラックホールは放射によってやがて消滅することを示した「ホーキング放射」の話であり、2つめが宇宙の始まりの話をアップグレードさせた「宇宙無境界仮説」なのです。\par
そして、このような経緯から、ホーキングは「量子宇宙論」という分野を創始したのでした。\par

\subsection{ホーキング放射}
ホーキングはまず、ブラックホールを使って量子論の適用を試みました。すると、その結果は驚くべきものでした。\par
物質を吸い込むだけで何も逃がさないと考えられていたブラックホールは、量子効果によって「放射」を行っていたことがわかったのです。つまり、光をも吸い込む真っ黒だと思っていたブラックホールは、ちょっぴり灰色だったわけです。\par
そしてこの放射によりエネルギーが逃げるため、ブラックホールは長い時間が経つうちにどんどん小さくなって、ついには「蒸発」してしまうであろうことがわかりました。\par
もう少し詳しくいうと、通常、星が崩壊してブラックホールとなるには、その星の質量が、少なくとも太陽の30倍は必要です。そのくらいの重い星が崩壊して縮んでいくと、どんどん冷えていき、最終的には絶対零度の100分の1度くらい上、というほど冷たくなります。現在の宇宙の温度が絶対零度の2.7度くらい上だとわかっているので、普通のブラックホールは、その周りの宇宙の温度に比べて圧倒的に冷たいのです。\par
こうなると、ブラックホールは冷えた体でお風呂に入っているようなものなので、自分でも熱を放射しているものの、その温度差によって周囲から吸収する熱のほうが大きい状態となります。それによって、周りからエネルギーを一方的に吸収しているように見えるのです。\par
ただし、宇宙は膨張を続けており、それに伴って宇宙の温度は下がっているので、遠い将来、宇宙の温度がブラックホールの温度を下回った時この関係は逆転し、ブラックホールは周囲の空間に熱とエネルギーを放出するようになり、少しずつ質量を失い、軽くなっていき、やがて質量がゼロになり「蒸発」してしまうのです。\par
ホーキングは、量子論を基にして、このように予言したのでした。\par
全てを吸い込むブラックホールが放射しているという論文は、当初、かなり驚きと不信をもって迎えられたようですが、現在ではおおむね合意を得た考え方となっています。

\subsection{宇宙無境界仮説}
いよいよホーキングは、宇宙の始まりに量子論をあてはめていきます。\par
ここから圧倒的に理解しにくくなっていくのですが、この仮説を説明するにあたって重要な概念である「虚時間」について簡単に説明したいと思います。(自分も深く理解できているわけではないのですが…。)\par
まず、私たちがいつも体験している時間を「実時間」とします。そして、この「虚時間」というのは、その実時間と直交して流れている時間であり、「虚数で測定される時間」です。\par
しかし、物理的に実際の時間軸がそのようになっているわけではなく、あくまで数学的な都合でこのように説明されているだけのものです。\par
このような時間を考えることによって、宇宙の特異点を計算すると、実時間(一般的に私たちが感じている時間)だけだと、大きさのない点であり、そこでは温度や密度などの物理量が無限大になるせいで、物理理論が破綻してしまいますが、虚時間を考えると、この宇宙の始まりは1点ではなく「丸くなって」特異点は見えなくなります。\par
「時間を虚数で測る」というこの発想は、かなりぶっとんだ話ですし、それによって「うまく計算が成立する」と言われたって、「だから何?」と思う人も多いかもしれません。また、これは単なる数学的なトリックに過ぎず、現実世界と何の関わりもない、それこそ「机上の空論なんじゃないのか」と言いたくなる人もいるかもしれませんが、ホーキングは実証論者なので、実際に虚時間のような時間があるかなどは気にしません。彼にとって問題なのは実時間だけで計算すると破綻してしまうが、虚時間を用いると計算が成立する」ということのみなのです。\par
これによって、宇宙の始まりは、尖った点ではなく、丸いキャップのようなイメージとなります。\par
つまり、宇宙の「始まり」は、尖った1点からボーンと宇宙に広がったというよりは、丸い状態からスタートした。そこから宇宙が爆発して広がっていった、というイメージになるのです。\par
虚時間の導入によって「始まり」が丸くなったというのは、宇宙の歴史が南極点から始まることのようなものだ、とホーキングは説明します。\par
「宇宙の始まりの前に何があったの?」という質問は、虚時間を「始まり」に導入することで、ますます意味を持たなくなります。それは、南極点に立った状態で「ここより南はどっち?」と聞いているようなものだからです。これ以上南はありませんが、地球はちゃんとありますし、南極点に立った旅人は、移動を続けることが可能です。\par
南極点を通り過ぎたら、いつの間にか北に向かって歩いているだけ。過去に遡って行き止まりにぶつかるとか、深い穴に落ちるとか、そういうことはなくて、ただ滑らかな時空のなかの一点を通り過ぎるだけなのです。\par
特異点定理でイメージされたような尖った特異点、「それより前はない」というような一点は、時間軸のなかに存在しないことになりました。ホーキングは、虚時間を採り入れることで、宇宙の始まりから特異点を消滅させてしまったのです。\par
その結果、時間が始まる前、始まる後という境界線も消えることとなりました。特異点のなくなった虚時間の宇宙のことをホーキングは「宇宙の境界条件は境界がないことだ」と表現しました。そこで、これを「無境界仮説」と名付けたのです。\par
この仮説については、専門家の物理学者たちも、実際のところ、このように虚時間と実時間を用いて宇宙の始まりを説明しようとすることに意見の一致を見せているわけではありません。最新の量子重力理論の動向を見ていると、どうやらホーキングの考え方は正しい、ということになりそうですが、実際のところはまだ観測による判定は明確には下っていない状態です。

\newpage

\section{物理学の考察について}


\subsection{物理学者}
物理学者たちは、数学を用いてSFを書いているようなもので、いろいろな手段によってより現実味のあるSFを書くことに全力を傾けています。\par
そして、物理学者のなかで仮説を認めさせるには、その仮説が正しいだけでなく美しい必要があります。「宇宙はこのような仕組みで動いている」という数式を提案したとして、現実にはさまざまな条件が発生しますから、その通りにはいかないこともあります。そのとき、変数として、いろいろな計算処理を数式に付け加えることで補正をかけ、現実に数式を合わせていくことは不可能ではありませんが、それは、後付けのようで美しくありません。\par
そうではなく、膨大なデータをシンプルな数式で一括して表現することが「美しい」のです。\par
ホーキング博士は理論物理学の人ですから、宇宙の始まりをどうやって説明できるかということを考えたとき、いかにシンプルな方程式で表現するか、にこだわります。そこには彼のアイデアが必ず盛り込まれています。アインシュタイン方程式と量子力学の基礎方程式を組み合わせるというのは、宇宙論者ならだれもが夢見る新しいことをするための「次の一歩」であり、それを実現したのがホーキングなのでした。アインシュタインの重力理論を量子力学と組み合わせるので、これを「量子重力理論」といいます。\par
つまり、ホーキングは量子重力理論の分野に先鞭を付けた人、という言い方もできるわけです。\par

また、物理学者たちは、観測や数学的根拠に基づいて有力な仮説を立てて、互いにつぶしあいます。\par
それ故に、ここで紹介したホーキング博士の仮説や定理も覆る可能性もあります。ですが、現在において有力な仮説の一つとして、このホーキングの考える「宇宙の始まり」はあります。\par






\newpage

%% 参考文献
\begin{thebibliography}{99}
  \item 竹内 薫,『ホーキング博士人類と宇宙の未来地図』,宝島社,2018.
  \item 須藤 靖,『不自然な宇宙』,講談社,2019.
  \item @vuccaken, 物科研HP, \url{https://vuccaken.github.io}, 2019.
  % \bibitem{キー1} 著者, 本やページの名前, (URL), 出版社, 出版年.
\end{thebibliography}


\end{document} % - - - - - - - - - - - - - - - - - - - - -
%
% ファイトだよ!
%