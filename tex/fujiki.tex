%
%   TEMPLATE
%   コピペ用
%

\documentclass[10pt,b5paper,papersize,dvipdfmx]{jsbook}

\usepackage{vuccaken}
\usepackage{vuccaken2019}

% スタイルファイルの読み込みや自作マクロは、
% 最終的には vuccaken2019.sty の中に書いてください。
% とりあえずはここに書いてもらって構いません。


\begin{document} % - - - - - - - 以下本文 - - - - - - - - - -

\mokuji{2} % 目次出力

% - - - - - - - - - - - - - - - - - - - - - - - - %
\kaishititle%
  {\LaTeX テンプレート(会誌原稿用)}% title
  {物理科学科1回生}% 所属
  {ふじき}% name
% - - - - - - - - - - - - - - - - - - - - - - - - %

%
\section*{はじめに}
今年度から物理科学研究会に入部した物理科1回生の藤木です。\par
今回の会誌では自分は宇宙についてホーキング博士の考えを紹介しながら書いていこうと思っています。

\newpage
%
\section{第1章 宇宙に関する疑問}
皆様は宇宙に対する疑問といわれるとどのようなものを思い浮かべるでしょうか。\par
figureのパスには注意してください。

\newpage


\section{第2章 ホーキング博士について}

\subsection{ホーキング博士}
ホーキング博士は記憶に新しい2018年3月14日に享年76歳でこの世を去ったイギリスの理論物理学者で、
本名はスティーブン・ウィリアム・ホーキングといい、彼は生前その業績とALS(筋萎縮性側索硬化症)
を患って車椅子に乗っていたことから\par
「車椅子のニュートン」と呼ばれていました。\par
ホーキング博士の業績については後の章でふれるのでこの章ではホーキング博士の人となりについて書いていこうと思います。\par
\,\par
\subsection{ホーキングの生い立ち}
彼は1942年1月8日にイギリスのオックスフォードに生まれました。当時、イギリスは第二次世界大戦のまっただなかで、
彼の両親はロンドン北部から疎開してオックスフォードに暮らしていたそうです。
彼が8歳の時、一家はロンドンの北にあるセント・アルバンズへ引っ越し、ホーキングはそこにある学校に通い始めました。\par
その後、父親と同じカレッジであるオックスフォード大学のユニバーシティー・カレッジに進学し、物理学を専攻したホーキングは1962年に
「自然科学」で学士号を取得します。\par
オックスフォード大学を卒業した彼はケンブリッジ大学の博士課程に進み、1965年、宇宙論の研究で博士号をもらいました。そして、その後もケンブリッジ大学に
研究フェローとして残り、アインシュタインの重力理論と量子論の研究を続けました。\par
1974年には史上最年少で王立協会のフェローに選出され、1977年にはケンブリッジ大学の重力理論講座の教授に就任されます。さらに彼は1979年には
37歳でかつてニュートンが在籍したというルーカス職数学教授に就任します。\par
このような華々しい経歴の裏でホーキングは21歳で筋萎縮性側索硬化症(ALS:Amyotrophic Lateral Sclerosis)の診断を受けており、それ以来”不治の病”と戦い続けることになってしまいました。\par
この筋萎縮性側索硬化症は治療のための有効な治療法は現在も確立されていない難病で、病気が進むに従って、手足をはじめ、体が自由に動かせなくなり、やがて話すこと、食べることはおろか、呼吸することも難しくなっていきます。\par
そんな難病を患って余命いくばくもないといわれてもなおホーキングは絶望するのではなく前へ進み続けました。
\newpage

\subsection{病を得てすべてがボーナスになる}
ALSという難病だと診断されても絶望せず、前に進み続ける精神を与えたのは、病院で目撃した白血病の少年だったといいます。\par
自伝でも多くを語ってはいませんが、余命いくばくもないと知らされた衝撃のなか、ホーキングは彼の死を目撃しました。
自分は最悪の運命だと思っていたが、そうではなかった。自分よりもっと不幸な人がいる。
そんなふうに思い直して、ホーキングは自分を憐れみたくなりそうになったときには




\newpage

%% 参考文献
\begin{thebibliography}{99}
  \item 著者, 本やページの名前, (URL), 出版社, 出版年.
  \item (複数ある場合は追加)
  \item @vuccaken, 物科研HP, \url{https://vuccaken.github.io}, 2019.
  % \bibitem{キー1} 著者, 本やページの名前, (URL), 出版社, 出版年.
\end{thebibliography}


\end{document} % - - - - - - - - - - - - - - - - - - - - -
%
% ファイトだよ!
%