%
%   TEMPLATE
%   コピペ用
%

\documentclass[10pt,b5paper,papersize,dvipdfmx]{jsbook}

\usepackage{vuccaken}
\usepackage{vuccaken2019}

% スタイルファイルの読み込みや自作マクロは、
% 最終的には vuccaken2019.sty の中に書いてください。
% とりあえずはここに書いてもらって構いません。


\begin{document} % - - - - - - - 以下本文 - - - - - - - - - -

\mokuji{2} % 目次出力

% - - - - - - - - - - - - - - - - - - - - - - - - %
\kaishititle%
  {\LaTeX テンプレート(会誌原稿用)}% title
  {物理科学科1回生}% 所属
  {ふじき}% name
% - - - - - - - - - - - - - - - - - - - - - - - - %

%
\section*{はじめに}
今年度から物理科学研究会に入部した物理科1回生の藤木です。\par
今回の会誌では自分は宇宙についてホーキング博士の考えを紹介しながら書いていこうと思っています。

\newpage
%
\section{宇宙に関する疑問}
皆様は宇宙に対する疑問といわれるとどのようなものを思い浮かべるでしょうか。\par
figureのパスには注意してください。

\newpage


\section{ ホーキング博士について}

\subsection{ホーキング博士}
ホーキング博士は記憶に新しい2018年3月14日に享年76歳でこの世を去ったイギリスの理論物理学者で、
本名はスティーブン・ウィリアム・ホーキングといい、彼は生前その業績とALS(筋萎縮性側索硬化症)
を患って車椅子に乗っていたことから「車椅子のニュートン」と呼ばれていました。\par
ホーキング博士の業績については後のセクションでふれるのでこのセクションではホーキング博士の人となりについて書いていこうと思います。\par

\subsection{ホーキングの生い立ち}
彼は1942年1月8日にイギリスのオックスフォードに生まれました。当時、イギリスは第二次世界大戦のまっただなかで、
彼の両親はロンドン北部から疎開してオックスフォードに暮らしていたそうです。
彼が8歳の時、一家はロンドンの北にあるセント・アルバンズへ引っ越し、ホーキングはそこにある学校に通い始めました。\par
その後、父親と同じカレッジであるオックスフォード大学のユニバーシティー・カレッジに進学し、物理学を専攻したホーキングは1962年に
「自然科学」で学士号を取得します。\par
オックスフォード大学を卒業した彼はケンブリッジ大学の博士課程に進み、1965年、宇宙論の研究で博士号をもらいました。そして、その後もケンブリッジ大学に
研究フェローとして残り、アインシュタインの重力理論と量子論の研究を続けました。\par
1974年には史上最年少で王立協会のフェローに選出され、1977年にはケンブリッジ大学の重力理論講座の教授に就任されます。さらに彼は1979年には
37歳でかつてニュートンが在籍したというルーカス職数学教授に就任します。\par
しかし、このような華々しい経歴の裏でホーキングは21歳で筋萎縮性側索硬化症(ALS: Amyotrophic Lateral Sclerosis)の診断を受けており、それ以来“
不治の病”と戦い続けることになってしまいました。\par
この筋萎縮性側索硬化症は治療のための有効な治療法は現在も確立されていない難病で、病気が進むに従って、手足をはじめ、体が自由に動かせなくなり、やがて話すこと、食べることはおろか、呼吸することも難しくなっていきます。\par
そんな難病を患って余命いくばくもないといわれてもなおホーキングは絶望するのではなく前へ進み続けました。
\newpage

\subsection{病を得てすべてがボーナスになる}
ALSという難病だと診断されても絶望せず、前に進み続ける精神を与えたのは、病院で目撃した白血病の少年だったといいます。\par
自伝でも多くを語ってはいませんが、余命いくばくもないと知らされた衝撃のなか、ホーキングは彼の死を目撃しました。
自分は最悪の運命だと思っていたが、そうではなかった。自分よりもっと不幸な人がいる。
そんなふうに思い直して、ホーキングは自分を憐れみたくなりそうになったときは、いつもその少年のことを思い出して、
自分に鞭を入れていたのでした。\par
「死刑宣告を受けたが、まだ執行猶予期間中。その間に、少しでも善いことをしたい」\,
というのがホーキングの願いだったようです。\par
ホーキングにとって、ALSを宣告されて以来、すべての時間は「ボーナス」でした。いつ死んでもおかしくないという身であれば、
人生に期待は禁物でしょう。そんな徹底した悲観論から生まれる今日という日へのボーナス感。それが50年続いたからこそ、ホーキングは思考において
誰よりもエネルギッシュに、遠くへ行くことができたのかもしれません。\par
もし私たちがホーキングと同じようにすべてをラッキーに感じて生きることができたら。
ホーキングの業績は、人間のポテンシャルを独特のかたちでみせてくれているものだ、とも言えるのではないでしょうか。\par
彼は生前、2004年12月12日ニューヨークタイムズにこのような言葉を残しています。\par
\begin{quote}
「My expectations were reduced to zero when I was 21. Everything since then has been a bonus.」\par
(私の期待は21歳の時にゼロまで落とされた。それ以来、すべてが私にとってボーナスなんだ。)\par
\end{quote}

\subsection{宇宙に「夢」を語らない実証論者}
宇宙論者というと、頭のいいロマンチストというイメージもあるかもしれませんが、ホーキングは違います。彼は、徹底的な実証論者であり、無神論者でもありました。\par
この実証論者とは、現象が起こった時に、何がどのように働いてその現象を起こしたのかについて考えずにその現象にのみ焦点を当て
「何が起きたのか」という結果にだけ関心を寄せる論者です。宇宙論者はざっくり言うとこの実証論者と実在論者に分かれていて、
後者の実在論者は反対に「現象の裏には何が存在するのか」に関心を寄せます。\par
\newpage

ホーキング博士はバリバリの実証論者で研究手法のスタンスとしてはかなりビジネスライクで、
結果だけを重要視して「モノで確かめたい」という衝動を抱くことなく進んでいきます。
考えても仕方ないものは考えない。現実的にいきましょうよ、と。理論的に整合しているなら、それで納得なのです。\par
また、彼は無神論者で後に紹介する無境界仮説において半ば神の存在を否定したことによって1981年に開かれたヴァチカン会議で当時の法王に「あんまり行き過ぎないように」とクギを刺されたといいます。(直接的にいわれたわけではありませんが。)\par
そんな無神論者であったホーキング博士にとっては「あの世」は存在しないわけで、彼は自分の存在をコンピューターに例えていて、死ぬ=脳が活動お停止することであり、つまりそれは壊れたコンピューターのようなものだ、と述べています。コンピューターは壊れたらそれっきりです。
自分も死んだら同じ、というのは、非常に潔い人生観と言えるでしょう。\par
人間誰しも死は不安なものですから、死んだあとも自分の居場所はあってほしい、天国があってほしい、というふうに考えたくなります。
自分が死んで、肉体が朽ち果てて、脳が機能を停止した後も、もしかしたら心とか魂とか呼ばれるものは残っていて、天国に行って安らぎを得られたら嬉しい、みたいに考える。いわば「死後の世界」を保険として引き合いに出しているのです。\par
しかし、ホーキングはそこを無神論者として、一切否定します。自分の死についてかなりドライに語っているのは、死そのものが、もうあまり怖くなくなっていたからではないかな、と想像します。ALSを患って以来、彼は他の誰よりも自らの死について深く頻繁に向き合ってきたはずです。
そのように自分の死や自分が死んだ後のことにさまざまな角度から考えを巡らせ尽くしていたからこそホーキング博士は動揺したり美化したりすることなく、死と向き合うことができていたのではないでしょうか。\par



\newpage

\section{ホーキング博士と宇宙}

\subsection{車椅子のニュートン}
先ほどのセクションでふれたように生前ホーキング博士は車椅子のニュートンと呼ばれていました。\par
そんな彼の科学者としての業績は、どのようなものだったのでしょうか?その答えは、まさにこの呼び名が如実に表しています。\par
つまり彼は、ニュートン級の物理学者でした。\par
ニュートンのように、物理学者たちの、そして人々の世界観を変えるほどの成果を挙げたのです。\par
では、アイザック・ニュートン(1642~1727)以前と以後では、物理学は何が違ったのでしょう。それは、ニュートン力学があったかなかったか、です。\par
ニュートンが登場するする以前にも力学に関する法則はあれこれ発見されていましたが、それらは個々の現象を説明する法則として、バラバラに存在しているにすぎませんでした。\par
例えば、ガリレオは真空状態であれば落下する物体の速度はどれも等しいという「落体の法則」を発見していましたし、ケプラーは、惑星の運動に関する「ケプラーの法則」を発見していました。ただし、それらは、特定の事象を説明するためだけに用いられる法則だったわけです。\par
ニュートンは、地上における落体の法則と宇宙における惑星の運動の法則が、実は同じ運動法則で説明できることを発見しました。それがニュートンの運動三法則と万有引力です。これにより、地上と宇宙は同じ法則で動いている、という認識が広まりました。\par
地球と宇宙は別ものだという人々の見方、考え方を完全に変えてしまったわけです。\par
それまで地上を支配する法則と天上界(宇宙はこう呼ばれていました)を支配する法則は違う、と人々は思っていましたが、実は同じなんだ、と。ガリレオとケプラーは同じことを言っているんだということが、ニュートン力学の確立によって示されたのです。\par
このおかげで、後の時代に生きる我々は、地上における重力の法則と惑星の動きはおなじ物理法則の支配下にあるという前提のもと、技術や文明を発展させ、ついには「はやぶさ」のような探査ロケットなどをつくることができるようになりました。\par
地上には地上の法則があり、天上界には天上界の法則がある。そういうかつて誰しも当たり前に感じていた世界観、宇宙観をがらりと変えてしまった。\par
「ニュートン級」の仕事といえば、このように人々のパラダイム(時代ごとの支配的な考え方・味方)を根っこから覆すような業績といえます。\par





\newpage

%% 参考文献
\begin{thebibliography}{99}
  \item 竹内 薫,『ホーキング博士人類と宇宙の未来地図』,宝島社,2018.
  \item 須藤 靖,『不自然な宇宙』,講談社,2019.
  \item @vuccaken, 物科研HP, \url{https://vuccaken.github.io}, 2019.
  % \bibitem{キー1} 著者, 本やページの名前, (URL), 出版社, 出版年.
\end{thebibliography}


\end{document} % - - - - - - - - - - - - - - - - - - - - -
%
% ファイトだよ!
%