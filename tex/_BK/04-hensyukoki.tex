%
%     編集後記
%

\documentclass[10pt,b5paper,papersize,dvipdfmx]{jsbook}

\usepackage{vuccaken}
\usepackage{vuccaken2019}

% --------------------------------------
\begin{document}

\begin{preface}{編集後記}
        {会計}% position
        {物理科学科4回生}% belong
        {\vname{中山}{敦貴}}% name
        {令和元年11月29日}% date
%%
はじめての人ははじめまして、はじめてじゃない人はお久しぶり、今年も編集を担当しました4回生(老害)の中山です。
まずは、会誌を手に取っていただいたことに感謝申し上げます。

今年も例年のごとく締切を破りに破り、学園祭当日2日前になってようやく会誌の内容をコンプリートすることができました。
みなさん、お疲れ様でした(原稿を落とした2名は除く)。
校正にも素直に従っていただきありがとうございました。

紆余曲折ありましたが、今回の表紙イラストは1回生の藤原君に描いていただきました。
藤原君曰く、立命館のオリジナルのRのロゴは黄金比だが、彼が描いたRでは白銀比が使われているそうです。謎のネギは僕が原因ですが、詳しくは述べないことにしましょう。


さて、ここからが本題ですが、今年の会誌も \LaTeX を使って作成していることはお分かりかと思います。
クラスは jsbook.cls を使用していますが、各章のタイトルやそれに伴う目次などは、昨年度と同様に新しくコマンドを定義して出力しています(名前を入れるため)。

結局、ディレクトリ構造以下のような感じで落ち着きました。
\begin{verbatim}
    kaishi2019/
      ├─ sty/
      │    ├─ vuccaken.sty      % 各章のタイトル出力コマンドなどを定義
      │    └─ vuccaken2019.sty  % 共通プリアンブル
      ├─ tex/
      │    ├─ _BK/              % 表紙や巻頭言など
      │    ├─ dir01/            % 会員1
      │    └─ dir02/            % 会員2
      │         ├─ img/         % 画像
      │         └─ file02.tex   % 各章の本文
      └─ marge.tex               % ここで各章のtexをマージする
\end{verbatim}

注意点は、サブディレクトリ以下のstyファイルにはパスを通さないといけない
\footnote{
  環境変数\texttt{TEXINPUTS}に\texttt{./sty/}や\texttt{../../sty/}のパスを追加する必要があります。
}
のと、\verb|\begin{document}~\end{document}|内だけを読み込むように\verb|\input|コマンドを改良した\texttt{docmute.sty}を使用したりと、いろいろ面倒なことがあります。
画像などを指定するパスも、最終的にtexファイルをinputする\texttt{marge.tex}から見た相対パスにしなければいけないので、\verb|\input|をさらに改造した\verb|\vInputTeX|を定義しました。

\begin{verbatim}
  \newcommand{\vInputTeX}{\@dblarg\v@InputTeX}
  \def\v@InputTeX[#1]#2{             % 引数が1つだけなら #1 = #2
    \def\pathToFiles{./tex/#1/}      % 自作。オプションから引数をとる
    \def\input@path{{\pathToFiles}}  % \inputなどのソース探索パスを追加
    \graphicspath{{\pathToFiles}}    % \includegraphicsのソース探索パス
    \input{\pathToFiles#2}}          % documute.styによって改良済み
\end{verbatim}

上のディレクトリ構造の例だと\verb|\vInputTeX[dir02]{file02}|のようにして使用します。他の人と画像のファイル名が被っても大丈夫なようになっています。
\verb|\label|に関しては他の人と被ってはいけませんが、\texttt{marge.tex}をタイプセットした際に multipled label があれば警告(alert)を吐いてくれるので、手動で直せばいいかなと思います。

また、今年からは会誌をGitHubで管理することにしました
\footnote{
  2019年度会誌のrepository: \url{https://github.com/vuccaken/kaishi2019}
}。
去年までは、変更があるたびslackにuploadするという面倒臭いことをしていたので、開発はかなりスムーズになったと思います(何度か進捗が消滅するミスはありましたが)。

あとは、弊研究会のホームページのリニューアル
\footnote{
  新しいホームページ:\url{https://vuccaken.github.io}
}
など、最後にいろいろ置き土産をしておいたので、これで無事引退させていただけるかと思います
\footnote{修士で立命館に再入学しますが。}
。
一時期は会員が少なく廃部の危機にも見舞われましたが、なんとか立て直すことができてよかったです。

1949年に核物理研究会として設立されてからちょうど70年、OBの方のご協力もあり、無事に令和の時代を迎えられたことは非常に嬉しく思います。
今後も末長く存続していくことと、いつか弊研究会員からノーベル賞受賞者が出ることを祈りつつPCを閉じることにします。
最後まで読んでいただいた方、関係者の皆様には改めて感謝申し上げます。
サンクス☆ベリマッチ!

\vspace{1zw}
\noindent{\vEmphFont% \bfseries
  P.S. 会員へ
}\par
来年度は週5でGitと\LaTeX のゼミをやります。お覚悟を。

\end{preface}

\end{document}

%
% お疲れさまです
%