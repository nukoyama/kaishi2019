%
%    comment
%


\documentclass[10pt,b5paper,papersize,dvipdfmx]{jsbook}

\usepackage{vuccaken}
\usepackage{vuccaken2019}
\usepackage{v-cmt}

\begin{document}

% - - - - - - - - - - - - - - - - - - - - - - - - %
\kaishititle%
  {句読点の話}% title
  {校正科学科}% 所属
  {nkym}% name
% - - - - - - - - - - - - - - - - - - - - - - - - %

\section*{はじめに}
和文中の句読点は\textcolor{red}{全角}でお願いします。
理由は以下を見て貰えばよいかと。

反論があればしてください。

\section{例}

\subsubsection{理由1}
そもそも、和文中に欧文の記号を使うのはおかしい。
句読点以外に、半角の(括弧)とかも。

\subsubsection{理由2}
半角句読点(カンマ、ピリオド)の後は半角スペースが必須だが、忘れやすい。

\subsubsection{理由3}
和文中の半角句読点は、\LaTeX がうまく処理してくれない。

\begin{table}[H]
  \centering
  % \caption{}
  \begin{tabular}{l|ll}
    \hline
    & 句読点 & 例 \\
    \hline
    OK & 全角(、。) & 「あ、こんちわ。」、とか、『おやすみ』。(など。)。\\
    OK & 全角(,.) & 「あ,こんちわ.」,とか,『おやすみ』.(など.).\\
    BAD & 半角(, . ) & 「あ, こんちわ. 」, とか, 『おやすみ』. (など. ).\\
    \hline
  \end{tabular}
\end{table}



\end{document}